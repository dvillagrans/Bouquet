\documentclass[12pt]{report}

% ======= Paquetes base =======
\usepackage[utf8]{inputenc}
\usepackage[T1]{fontenc}
\usepackage[spanish,es-nodecimaldot]{babel}
\usepackage{geometry}
\geometry{letterpaper, top=2.5cm, bottom=2.5cm, left=3cm, right=2.5cm}
\usepackage{setspace}
\usepackage{titlesec}
\usepackage{enumitem}
\usepackage{booktabs}
\usepackage{array}
\usepackage{longtable}
\usepackage{tabularx}
\usepackage{graphicx}
\usepackage{lscape}
\usepackage{pgfgantt}      % Diagramas de Gantt
\usepackage{pdflscape}     % (opcional) página apaisada
\usepackage{xcolor}        % (opcional) colores de barras
\usepackage{csquotes}
\usepackage[backend=biber]{biblatex}
\addbibresource{referencias.bib}

% ======= Times New Roman =======
\usepackage{mathptmx}

% estilos opcionales para que se vea pro
\ganttset{
  calendar week text={},
  bar/.append style = {fill=black!10},
  group/.append style = {fill=black!20},
  milestone/.append style = {fill=black},
  progress label text = {},
  link/.style = {->, thick},
}

% ======= Interlineado y párrafo (E7) =======
\setstretch{1.0} % sencillo
\setlength{\parindent}{1.25cm}
\setlength{\parskip}{0pt}
% Configurar tamaño de fuente base a 12pt
\renewcommand{\normalsize}{\fontsize{12}{14.4}\selectfont}
\normalsize

% ======= Títulos numerados 12 pt (E6) =======
\titleformat{\chapter}{\bfseries\fontsize{12}{14.4}\selectfont}{\thechapter.}{1em}{}
\titlespacing*{\chapter}{0pt}{20pt}{15pt}
\titleformat{\section}{\bfseries\fontsize{12}{14.4}\selectfont}{\thesection.}{0.6em}{}
\titleformat{\subsection}{\bfseries\fontsize{12}{14.4}\selectfont}{\thesubsection.}{0.6em}{}

% ======= Comandos de portada (E1–E5) =======
\newcommand{\TituloTT}[1]{\begin{center}\textbf{\Large #1}\end{center}} % 14 pt, negrita, centrado
\newcommand{\Proposito}[1]{\begin{center}#1\end{center}}
\newcommand{\Registro}[1]{\begin{center}\textbf{\textit{#1}}\end{center}} % 12 pt, negrita+cursiva, centrado
\newcommand{\NotaPequena}[1]{{\normalsize \textbf{#1}}}

% ======= Tablas compactas =======
\newcolumntype{L}[1]{>{\raggedright\arraybackslash}p{#1}}
\newcolumntype{C}[1]{>{\centering\arraybackslash}p{#1}}
\newcolumntype{R}[1]{>{\raggedleft\arraybackslash}p{#1}}

% ======= Configuración para tablas y figuras (10pt) =======
\newcommand{\tablefont}{\fontsize{10}{12}\selectfont}
\newcommand{\captionfont}{\fontsize{10}{12}\selectfont}

% ======= Hyperref al final =======
\usepackage{hyperref}
\hypersetup{colorlinks=true, linkcolor=black, urlcolor=blue, citecolor=black}

\begin{document}

% =======================
% PORTADA / CARÁTULA
% =======================
\thispagestyle{empty}

\begin{center}
% \includegraphics[width=3cm]{logo_ipn.png} \hspace{2cm} \includegraphics[width=3cm]{logo_escom.png} \\[1cm]
\textbf{\Large INSTITUTO POLITÉCNICO NACIONAL} \\[0.5cm]
\textbf{\large ESCUELA SUPERIOR DE CÓMPUTO} \\[0.5cm]
\textbf{\large LICENCIATURA EN CIENCIA DE DATOS} \\[1.5cm]
\end{center}

\TituloTT{Sistema Omnicanal y Plataforma Progresiva para la Gestión de Mesas, Órdenes y Pagos en Restaurantes}

\Proposito{Desarrollo de una aplicación web progresiva que integra gestión de mesas virtuales, pedidos colaborativos y sistemas de pago flexibles para modernizar la experiencia gastronómica.} % (E2)

\Registro{Registro: \,[asignado por la CATT]} % (E3)

\vspace{1cm}
\begin{center}
\textbf{Trabajo Terminal No.:}\;[TT\#] \\[4mm]
\textbf{Alumnos:}\; Natalia Anaya Ramírez, Diego Villagrán Salazar \\[2mm]
\textbf{Director:}\; [Nombre del Director] \\[4mm]
\textbf{Correo de contacto:}\; \texttt{nataliaanayaa@gmail.com}
\end{center}

\newpage

% =======================
% RESUMEN Y PALABRAS CLAVE
% =======================
\section*{Resumen}
Este proyecto propone el desarrollo de una aplicación progresiva (PWA) para restaurantes, la cual combina acceso web y móvil en una sola plataforma. El sistema integra tres canales: (1) aplicación para usuarios, (2) panel de cocina y (3) panel administrativo administrativo. La característica principal es la gestión flexible de cuenta, donde cada usuario puede pagar su propio consumo, la de otros comensales, la cuenta completa, o realizar aportaciones a una mesa virtual compartida. De esta forma, se reorganiza la experiencia de pago y se centraliza información valiosa para análisis, trazabilidad y conciliaciones futuras.

\section*{Palabras clave}
Aplicaciones Progresivas; Restaurantes; Omnicanalidad; Sistemas Distribuidos; Inteligencia de Negocios.

\tableofcontents
\newpage

% =======================
% 7. CRONOGRAMA
% =======================
\chapter{Cronograma}

\begin{landscape}
\begin{figure}[ht]
\centering

\begin{ganttchart}[
  hgrid, vgrid,
  time slot format=isodate,
  x unit=0.5cm, y unit chart=0.7cm,
  bar height=0.5,
  group height=0.6,
  milestone height=0.6
]{2024-08-01}{2025-12-31}

  \gantttitle{Cronograma del Proyecto 2024--2025}{17} \\
  \gantttitlecalendar{year, month=name} \\

  % ===== Fase I: Diseño y Planificación =====
  \ganttgroup{Fase I: Diseño y Planificación}{2024-08-01}{2024-10-31} \\
  \ganttbar[bar/.append style={fill=green!60}]{Análisis de Requerimientos}{2024-08-01}{2024-09-30} \\
  \ganttbar[bar/.append style={fill=green!60}]{Diseño de Base de Datos}{2024-10-01}{2024-10-31} \\

  % ===== Fase II: Desarrollo Core =====
  \ganttgroup{Fase II: Desarrollo Core}{2024-11-01}{2025-10-31} \\
  \ganttbar[bar/.append style={fill=green!60}]{Mesas Virtuales y Cuentas}{2024-11-01}{2025-03-31} \\
  \ganttbar[bar/.append style={fill=orange!60}]{Módulo de Cocina}{2025-01-01}{2025-10-31} \\
  \ganttbar[bar/.append style={fill=orange!60}]{Sistema de Pagos}{2025-03-01}{2025-11-30} \\

  % ===== Fase III: Integración y Testing =====
  \ganttgroup{Fase III: Integración y Testing}{2025-09-01}{2025-12-31} \\
  \ganttbar[bar/.append style={fill=orange!60}]{Panel Administrativo}{2025-09-01}{2025-11-30} \\
  \ganttbar{Pruebas del Sistema}{2025-11-01}{2025-12-31} \\
  \ganttbar{Documentación Técnica}{2025-09-01}{2025-12-31} \\

  % ===== Fase IV: Finalización =====
  \ganttgroup{Fase IV: Finalización}{2025-12-01}{2025-12-31} \\
  \ganttbar{Validación y Entrega}{2025-12-01}{2025-12-31} \\

  % ===== Hitos =====
  \ganttmilestone{Diseño completo}{2024-10-31} \\
  \ganttmilestone{Core funcional}{2025-10-31} \\
  \ganttmilestone{Sistema completo}{2025-12-31} \\

  % Línea de "hoy"
  \ganttvrule[vrule/.style={draw=red!64, dashed, line width=1pt}]{Hoy}{2025-09-19}

\end{ganttchart}

\caption{Cronograma del proyecto con fechas actualizadas, fases, tareas e hitos principales}
\label{fig:cronograma-gantt}
\end{figure}
\end{landscape}

\vspace{0.5cm}

\textbf{Leyenda del Diagrama:}
\begin{itemize}[leftmargin=*, itemsep=2pt]
    \item \textcolor{green!60}{\rule{0.8cm}{0.25cm}} \textbf{Completado} - Tareas finalizadas al 100\%
    \item \textcolor{orange!60}{\rule{0.8cm}{0.25cm}} \textbf{En Progreso} - Tareas con avance parcial
    \item \textcolor{black!10}{\rule{0.8cm}{0.25cm}} \textbf{Planificado} - Tareas pendientes de iniciar
    \item \textcolor{black!20}{\rule{0.8cm}{0.25cm}} \textbf{Fases} - Agrupación lógica de actividades
    \item \textcolor{black}{\scalebox{1.2}{$\bullet$}} \textbf{Hitos} - Puntos críticos de entrega
    \item \textcolor{red!64}{\rule{0.8cm}{1pt}} \textbf{Línea Actual} - Momento presente del proyecto (Sept 2025)
\end{itemize}

\end{document}